%----------------------------------------------------------------------------------------
%   PACKAGES AND THEMES
%----------------------------------------------------------------------------------------

\documentclass[aspectratio=169]{beamer}

\mode<presentation> {

% The Beamer class comes with a number of default slide themes
% which change the colors and layouts of slides. Below this is a list
% of all the themes, uncomment each in turn to see what they look like.

%\usetheme{default}
%\usetheme{AnnArbor}
%\usetheme{Antibes}
%\usetheme{Bergen}
%\usetheme{Berkeley}
%\usetheme{Berlin}
%\usetheme{Boadilla}
%\usetheme{CambridgeUS}
%\usetheme{Copenhagen}
%\usetheme{Darmstadt}
%%%%%%%\usetheme{Dresden}
%\usetheme{Frankfurt}
%\usetheme{Goettingen}
%\usetheme{Hannover}
%\usetheme{Ilmenau}
%\usetheme{JuanLesPins}
\usetheme{Luebeck}
%\usetheme{Madrid} 
%\usetheme{Malmoe}
%\usetheme{Marburg}
%\usetheme{Montpellier}
%\usetheme{PaloAlto}
%\usetheme{Pittsburgh}
%\usetheme{Rochester}
%\usetheme{Singapore}
%\usetheme{Szeged}
%\usetheme{Warsaw}

% As well as themes, the Beamer class has a number of color themes
% for any slide theme. Uncomment each of these in turn to see how it
% changes the colors of your current slide theme.

%\usecolortheme{albatross}
%%%%%\usecolortheme{beaver}
%\usecolortheme{beetle}
%\usecolortheme{crane}
%\usecolortheme{dolphin}
%\usecolortheme{dove}
%\usecolortheme{fly}
%\usecolortheme{lily}
%\usecolortheme{orchid}
%\usecolortheme{rose}
%\usecolortheme{seagull}
%\usecolortheme{seahorse}
%\usecolortheme{whale}
%\usecolortheme{wolverine}
\usecolortheme{default}

%\setbeamertemplate{footline} % To remove the footer line in all slides uncomment this line
%\setbeamertemplate{footline}[page number] % To replace the footer line in all slides with a simple slide count uncomment this line

%\setbeamertemplate{navigation symbols}{} % To remove the navigation symbols from the bottom of all slides uncomment this line
}


\addtobeamertemplate{navigation symbols}{}{%
    \usebeamerfont{footline}%
    \usebeamercolor[fg]{footline}%
    \hspace{1em}%
    \insertframenumber/\inserttotalframenumber
}



\usepackage{graphicx} % Allows including images
\usepackage{booktabs} % Allows the use of \toprule, \midrule and \bottomrule in tables
\usepackage[utf8]{inputenc}

\usepackage{ragged2e}
\usepackage{lmodern}
\usepackage{array}
\usepackage[normalem]{ulem}
\usepackage{microtype}

\usepackage{graphicx}
\graphicspath{ {images/} }

\setbeamerfont{footnote}{size=\tiny}

\renewcommand{\figurename}{Figura}

%----------------------------------------------------------------------------------------
%   TITLE PAGE
%----------------------------------------------------------------------------------------


\title[Projeto de Visualização]{Técnica de Visualização Computacional Aplicada a Indicadores de Desenvolvimento Humano de Estados e Cidades do Brasil} 
% The short title appears at the bottom of every slide, the full title is only on the title page

\author[Leandro Ungari Cayres]{Leandro Ungari Cayres} % Your name

\institute[UNESP] % Your institution as it will appear on the bottom of every slide, may be shorthand to save space
{
Universidade Estadual Paulista \\ % Your institution for the title page
\medskip
\textit{leandroungari@gmail.com} % Your email address
}
\date{09 de Janeiro de 2018} % Date, can be changed to a custom date

\begin{document}

\begin{frame}
\titlepage % Print the title page as the first slide
\end{frame}

%----------------------------------------------------------------------------------------
%   PRESENTATION SLIDES
%----------------------------------------------------------------------------------------
\begin{frame}
\frametitle{Visão Geral}
\tableofcontents
\end{frame}

%==================
\begin{frame}
\frametitle{Introdução}
\justifying

O conceito de Desenvolvimento Humano objetiva mensurar o avanço de uma população não somente considerando os aspectos de âmbito econômico, mas também características sociais, culturais e políticas que influenciam diretamente na
qualidade da vida. 

A partir desse conceito, o Índice de Desenvolvimento Humano (IDH) foi criado com o intuito de contrapor outro indicador muito utilizado, o Produto Interno Bruto (PIB) per capita.



\end{frame}

%==================
\begin{frame}
\frametitle{Introdução}
\justifying

A utilização de representações visuais para a interpretação dos indicadores consiste em uma importante ferramenta de
interpretação, especialmente em um domínio de dados referentes a localidades específicas, possibilita o uso de mapas, e consequentemente, o agrupamento de regiões semelhantes para a identificações de características de uma região em detrimento das demais.

\end{frame}


%==================
\section{Conjunto de Dados}
\begin{frame}
\frametitle{Conjunto de Dados}
\justifying

\begin{columns}
\begin{column}{0.5\textwidth}


O conjunto de dados é proveniente do Atlas de Desenvolvimento Humano do PNUD Brasil, referente ao ano de 2013, o qual contém informações socioeconômicas de todos os estados e municípios brasileiros referentes aos anos de 1991, 2000 e 2010.

\end{column}

\begin{column}{0.4\textwidth}

\begin{figure}
\centering
\includegraphics[width=0.3\textwidth]{images/atlas.png}
\caption{Atlas de Desenvolvimento Humano no Brasil.}
\end{figure}


\end{column}
\end{columns}


\end{frame}

%==================
\begin{frame}
\frametitle{Índice de Desenvolvimento Humano}
\justifying

\begin{columns}

\begin{column}{0.5\textwidth}

O IDH foi criado no início da década de 90, para o Programa das Nações Unidas para o Desenvolvimento (PNUD), pelo conselheiro especial \textit{Mahbub ul Haq}, através da combinação de três componentes básicos do desenvolvimento humano: 

\end{column}

\begin{column}{0.5\textwidth}

\begin{figure}
\centering
\includegraphics[width=0.45\textwidth]{images/pnud.png}
\caption{Programa de Desenvolvimento das Nação Unidas.}
\end{figure}


\end{column}
\end{columns}

\end{frame}

%==================
\begin{frame}
\frametitle{Índice de Desenvolvimento Humano}
\justifying

\begin{itemize}
\item Longevidade, que também reflete, entre outras coisas, as condições de saúde da população; medida pela esperança de vida ao nascer;

\item Educação, medida por uma combinação da taxa de alfabetização de adultos e a taxa combinada de matrícula nos níveis de ensino fundamental, médio e superior;

\item Renda, medida pelo poder de compra da população, baseado no PIB per capita ajustado ao custo de vida local para torná-lo comparável entre países e regiões, através da metodologia conhecida como paridade do poder de compra.
\end{itemize}

\end{frame}

%==================
\section{Técnica de Visualização}


\begin{frame}
\frametitle{Técnica de Visualização}
\justifying

A técnica de visualização computacional escolhida foi a \textit{Choropleth Map}, ou Mapas Coropléticos, a qual consiste em um mapa temático quantitativo comum, em que as magnitudes das estatísticas baseadas em área (geralmente dados de atributo derivados) são retratadas através do preenchimento de cores, tons ou padrões, à medida que ocorrem dentro dos limites das áreas da unidade.

\end{frame}

\begin{frame}
\frametitle{Técnica de Visualização}
\justifying

\begin{figure}
\centering
\includegraphics[width=.45\textwidth]{images/boston-density.jpg}
\caption{Mapa de densidade demográfica de bairros da cidade de Boston.}
\end{figure}


\end{frame}

%==================
\begin{frame}
\frametitle{Técnica de Visualização}
\justifying

\begin{columns}

\begin{column}{0.5\textwidth}

\begin{itemize}

\item O termo Choropleth Map, foi primeiramente utilizado pelo geógrafo americano John Kirtland Wright em 1938, na sua obra \textit{Problems in Population Mapping}.

\item Por sua vez, o primeiro mapa coroplético conhecido foi construído pelo cartógrafo francês Pierre Charles François Dupin, em que são representados os índices de analfabetismo nas províncias francesas em 1826.

\end{itemize}

\end{column}

\begin{column}{0.5\textwidth}

\begin{figure}
\centering
\includegraphics[width=.6\textwidth]{images/analfabetismo-franca.jpg}
\caption{Mapa da taxa de analfabetismo na França em 1826.}
\end{figure}


\end{column}
\end{columns}


\end{frame}

%==================
\section{Estudos de Caso}
\begin{frame}
\frametitle{Estudos de Caso}
\justifying

Com o intuito de avaliar a utilização da técnica de visualização computacional \textit{Choropleth Map}, foram conduzidos os seguintes estudos de caso utilizando a base de dados do Atlas de Desenvolvimento Humano:

\begin{itemize}
\item Estudo de Caso I - Comparativo do Índice de Educação
\item Estudo de Caso II - Redimensionamento do Domínio de Dados
\item Estudo de Caso III - Evolução dos Índice de Desenvolvimento Humano
\item Estudo de Caso IV - Qualidade de Vida no Brasil
\end{itemize}


\end{frame}

\begin{frame}
\frametitle{Estudo de Caso I}
\justifying

\begin{figure}
\centering
\includegraphics[width=0.33\textwidth]{images/educacao-1991.png}
\caption{IDHM Educação 1991.}
\end{figure}


\end{frame}

\begin{frame}
\frametitle{Estudo de Caso I}
\justifying

\begin{figure}
\centering
\includegraphics[width=0.33\textwidth]{images/educacao-2000.png}
\caption{IDHM Educação 2000.}
\end{figure}


\end{frame}

\begin{frame}
\frametitle{Estudo de Caso I}
\justifying

\begin{figure}
\centering
\includegraphics[width=0.33\textwidth]{images/educacao-2010.png}
\caption{IDHM Educação 2010.}
\end{figure}


\end{frame}

\begin{frame}
\frametitle{Estudo de Caso II}
\justifying

\begin{figure}
\centering
\includegraphics[width=0.33\textwidth]{images/longevidade.png}
\caption{IDHM Longevidade 2010.}
\end{figure}

\end{frame}

\begin{frame}
\frametitle{Estudo de Caso II}
\justifying

\begin{figure}
\centering
\includegraphics[width=0.33\textwidth]{images/frequencia.png}
\caption{Gráfico de distribuição de ocorrências.}
\end{figure}

\end{frame}

\begin{frame}
\frametitle{Estudo de Caso II}
\justifying

\begin{figure}
\centering
\includegraphics[width=0.33\textwidth]{images/longevidade-ampliado.png}
\caption{IDHM Longevidade 2010 Normalizado.}
\end{figure}

\end{frame}

\begin{frame}
\frametitle{Estudo de Caso III}
\justifying


\begin{figure}
\centering
\includegraphics[width=0.3\textwidth]{images/idh-1991.png}
\caption{Índice de desenvolvimento humano em 1991.}
\end{figure}

\end{frame}

\begin{frame}
\frametitle{Estudo de Caso III}
\justifying


\begin{figure}
\centering
\includegraphics[width=0.3\textwidth]{images/idh-2010.png}
\caption{Índice de desenvolvimento humano em 1991.}
\end{figure}

\end{frame}

\begin{frame}
\frametitle{Estudo de Caso III}
\justifying


\begin{figure}
\centering
\includegraphics[width=0.3\textwidth]{images/evolucao.png}
\caption{Evolução do índice de desenvolvimento humano entre os anos de 1991 e 2010 (Normalizado).}
\end{figure}

\end{frame}

\begin{frame}
\frametitle{Estudo de Caso IV}
\justifying

\begin{figure}
\centering
\includegraphics[width=0.6\textwidth]{images/usa.png}
\end{figure}


\end{frame}

\begin{frame}
\frametitle{Estudo de Caso IV}
\justifying

\begin{figure}
\centering
\includegraphics[width=0.4\textwidth]{images/qualidade-de-vida.png}
\end{figure}


\end{frame}

%================
\begin{frame}
\frametitle{Considerações Finais}
\justifying





\end{frame}


\end{document}

\end{document}
              
            